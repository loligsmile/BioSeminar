
\section{Die Fragen der Denaturierung}
\subsection{Schmelzkurven - Denaturierung durchleuchtet}

\begin{frame}
\frametitle{Schmelzkurven}
\begin{block}{Schmelztemperatur $T_\text{m}$}
Temperatur bei der die Hälfte der DNA denaturiert ist.
\begin{itemize}
\item Abhängig von der DNA-Sequenz
\item Liegt zwischen 50-100 $^\circ$C
\end{itemize}
\end{block}
\centering
\vspace{0.5cm}
Experimentell aus Schmelzkurven bestimmbar.
\vspace{0.5cm}
\frametitle{Schmelzkurven}
\begin{block}{Schmelzkurve}
Relative Absorption $\theta$ bei meist 260 nm als Funktion der Temperatur.\\
\begin{itemize}
\item Kummulatives Schmelzprofil :  $1-\theta(T)$
\item Differentielles Schmelzprofil : $-\frac{\text{d}\theta}{\text{d}T}$
\end{itemize}
\end{block}
\end{frame}

\begin{frame}
\frametitle{Schmelzkurven}
\begin{minipage}{5.2cm}
\begin{figure}[h]
	\centering
		\includegraphics[width=5.0cm]{pic/Schmelz.jpg}
\end{figure}
\centering
\tiny\cite{GOT83}
\end{minipage}
\begin{minipage}{5.0cm}
\small{
\begin{itemize}
\item Schmelzkurven zeigen reproduzierbare Feinstruktur
%\item Einander folgende kooperative Teilübergänge
\item Schmelzregionen $\approx$ kBP denaturieren in Intervallen von 0,3-0,5 $^\circ$C
\item Schärfere Struktur für kurze DNA
\end{itemize}}
\end{minipage}
\begin{block}{}
\begin{itemize}
\item Feinstruktur durch Schmelzen kooperativer Schmelzregionen (Loops)
\item Öffnung eines Loops beeinflusst die Stabilität benachbarter Loops
\item Minimale Fluktuationen der BP verändern die Feinstruktur signifikant 
\end{itemize}
\end{block}
\centering
Die Öffnung von kooperativen Loops kann mittels Elektronenmikroskopie verfolgt werden.
\end{frame}


\subsection{Motivation}
\begin{frame}
\frametitle{Motivation}
\vspace{0.8cm}
\setstretch{1.5}
\begin{itemize}
\item Prozess der enzymatischen Denaturierung noch nicht verstanden
\item Verständnis thermischer Denaturierung wäre ein erster Schritt
\end{itemize}
\vspace{0.6cm}
\begin{figure}[]
	\centering
		\includegraphics[width=1\textwidth]{pic/Fragen.JPG}
\end{figure}
\end{frame}

\subsection{Fortschritt im Laufe der Zeit}
\begin{frame}
\frametitle{Fortschritt im Laufe der Zeit}
\setstretch{1.5}
\begin{tabbing}
 \textcolor{YellowGreen}{1953}$\qquad$\= Veröffentlichung Watson \& Crick zur DNA Doppelhelix\\
\textcolor{YellowGreen}{1960er}\> Beobachtung erster Schmelzkurven mit Feinstruktur\\
\textcolor{YellowGreen}{1970er}\> Goldene Ära der Schmelzstudien\\
\textcolor{YellowGreen}{1970}\> Erste DNA-Sequenzierungs Methoden\\
\textcolor{YellowGreen}{1977}\> Poland-Fixmann-Freire Algorithmus\\
\textcolor{YellowGreen}{1990er}\> Computertechnik und das WorldWideWeb\\
\textcolor{YellowGreen}{1990er}\> DNA Sequenzierung: schneller und günstiger\\
\textcolor{YellowGreen}{2000} \>Bestimmung der Ordnung des Phasenübergangs\\
\textcolor{YellowGreen}{2004} \>Erste Entschlüsselung des menschlichen Genoms
\end{tabbing}
\end{frame}