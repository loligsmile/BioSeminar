
\setstretch{1.0}

\subsection*{Strain 121}
\begin{frame}
 \frametitle{Strain 121}
\begin{itemize}
\item Faszinierender Mikroorganismus der Gattung Archae
\item Lebt 320 km tief im Pazifik
\item Gehört den Hyperthermophilen an
\item Hält den Weltrekord der Hitzebeständigkeit, lebt bei 121 $^\circ$ C
\item Theoretisch müsste die gesamte DNA denaturieren
\item Brachte bis dahin verwendete Sterilisatiosverfahren an ihre Grenzen
\end{itemize}
\centering
\begin{minipage}{3.5cm}
		\centering
\begin{figure}[h]
	\centering
		\includegraphics[width=2cm, angle=90]{pic/Strain121.png}
\end{figure}
	\cite{ST14}
\end{minipage}
\qquad
\begin{minipage}{4cm}
		\centering
\begin{figure}[h]
	\centering
		\includegraphics[width=3cm]{pic/autoklav.jpg}
\end{figure}
	\cite{AK14}
\end{minipage}
\end{frame}


\subsection*{PCR}
\begin{frame}
\frametitle{PCR - Polymerasekettenreaktion}
\begin{figure}[h]
	\centering
		\includegraphics[width=1\textwidth]{pic/PCR.png}
\end{figure}
\centering
	\cite{PCR14}

\begin{itemize}
\small{
\item Zur Vervielfältigung von DNA und RNA für z.B. DNA Sequenzierung
\item Durch hohe Temperaturen wird wirtseigene Polymerase meist zerstört
\item Verwendung der Taq-Polymerase um die Nukleotide zu verbinden}
\end{itemize}
\end{frame}

\subsection*{Taxonomie mittels GC-Gehalt}
\begin{frame}
\frametitle{Taxonomie mittels GC-Gehalt}
\begin{block}{Taxonomie}
Klassifikation von Lebewesen nach Kategorien
\end{block}
\centering
Bei Mikroorganismen gibt Körperbau und Stoffwechsel nur wenig Aufschluss
$\Rightarrow$ Klassifizierung auf Grund von DNA-Zusammensetzung
\vspace{0.3cm}
\begin{itemize}
\item Stapelwechselwirkung ist besonders groß wenn die Paare G--C und C--G aufeinander folgen
\item Damit ist GC reiche DNA thermisch stabiler
\item In den 1960ern wurde von Marmur \& Doty ein etwa linearer Zusammenhang zwischen $T_\text{m}$ und GC\% entdeckt.
\end{itemize}
Kennt man die Schmelztemperatur oder die DNA Sequenz, kann auf den GC-Gehalt geschlossen werden.\\

\vspace{0.5cm}
GC\%(Mensch) $\approx$ 41 \% $\qquad$ GC\%(Actinobacterium) $\approx$ 72 \%
\end{frame}




\begin{frame}
\frametitle{Schnittstellen der Singularitäten}
\begin{minipage}{5.5cm}
\begin{figure}
	\centering
		\includegraphics[width=1\textwidth]{pic/fisher.JPG}
\end{figure}
\centering \tiny \cite{FIS84}\\
\end{minipage}
\begin{minipage}{5cm}
\tiny

\begin{flushleft}

$1<\Psi<2\Rightarrow$ Kontinuierlicher Übergang\\
Aufgetragen sind die Singularitäten als Funktion von $\frac{1}{G_A(z)}$
\begin{itemize}
\item Singularität von $G_B(z)$ mit $\Psi=0$ ist $z_B=1/w$
\item $\nu^2 G_B(z)=\frac{1}{G_A(z)}$ durch $G(z)$
\item  $1-zu=\frac{1}{G_A(z)}$ durch Umstellen von $G_A(z)$
\end{itemize}
\end{flushleft}
\end{minipage}

  
\vspace{0.3cm}
\tiny
In die Funktion $\nu^2 G_B(z)=\frac{1}{G_A(z)}$ geht der Einfluss von $\Psi$ durch die Entwicklung von $G_B(z)$ um $wz\rightarrow1$ ein.\\
Es liegt ein kontinuierlicher Übergang für $u<u_c$ vor.
\end{frame}


%\begin{frame}
%\frametitle{Algorithmen}
%Beschleunigung der Berechnung der Schmelzkurven und Konfigurationswahrscheinlichkeiten.\\
%\vspace{0.4cm}
%\textbf{Azbel's Methode}
%\begin{itemize}
%\item Berücksichtig nur Grundszustandkonfigurationen $F_c$ minimal
%\item Grundzustandskonfiguration ändert sich stufenweise während der Temperaturerhöhung.
%\end{itemize}
%\vspace{0.2cm}
%\textbf{Poland-Fixman-Freire Methode}
%\begin{itemize}
%\item Bekanntester \& effizientester Algorithmus zur Berechnung von Schmelzprofilen
%\item Ohne Näherungen
%%%%%%%%%%%%%%%%%%%%%%%%%%%%%%%%%%%%%%%%%%%%%%%%%%%%%%%%%%%%%%%%%%%%%%%%%%%%%%%%%%%%%%%%%%%%%%%%%%%%%%%%%%%%%%%%%%%%%%%%%%%%%%\item Ergänzen..
%\end{itemize}
%\end{frame}