

\section{Einleitung - Ein wenig Biologie}
\subsection{Was ist Denaturierung?}
\begin{frame}
  \frametitle{Was ist Denaturierung?}
	\begin{block}{Denaturierung}
	\begin{itemize}
	\item Aufbruch von Bindungen
	\item Strukturelle Veränderung des Biomoleküls ab der Sekundärstruktur
	\item Veränderung/ Verlust der Funktion und spezifischer Eigenschaften
	\item Mögliche Ursachen: Temperaturerhöhung, Änderung des pH-Werts oder der Salzkonzentration, Strahlenschäden, Chemikalien
	\end{itemize}
	\end{block}
		\begin{minipage}{3.5cm}
		\centering
\begin{figure}[h]
	\centering
		\includegraphics[width=3cm]{pic/ei.jpg}
\end{figure}
	\cite{EI14}
\end{minipage}
\begin{minipage}{3.5cm}
		\centering
\begin{figure}[h]
	\centering
		\includegraphics[width=2cm]{pic/fieber.jpg}
\end{figure}
	\cite{TH14}
\end{minipage}
\begin{minipage}{3.5cm}
		\centering
\begin{figure}[h]
	\centering
		\includegraphics[width=3cm]{pic/quelle.jpg}
\end{figure}
	\cite{QU14}
\end{minipage}

\end{frame}


\begin{frame}
\frametitle{Schädigung? - Lebensnotwendig!}
\vspace{0.1cm}
\centering
\small Denaturierung ist ein notwendiger Bestandteil lebenserhaltender Prozesse.
Verwendung für Bio-Chemische-Technologien, wie PCR.
\vspace{0.3cm}
  
\centering
		\begin{minipage}{7cm}
		\centering
\begin{figure}[h]
	\centering
		\includegraphics[width=6.5cm]{pic/transkription.jpg}
\end{figure}
	\cite{TK14}
	Transkription
\end{minipage}
\begin{minipage}{5.0cm}
		\centering
\begin{figure}[h]
	\centering
		\includegraphics[width=4.5cm]{pic/Replikation.png}
\end{figure}
	\cite{RK14}
	Replikation
\end{minipage}
\end{frame}

\subsection{Struktur der DNA}
\begin{frame}
\frametitle{Struktur der DNA}
		\centering
\begin{figure}[h]
	\centering
		\includegraphics[width=10cm]{pic/Backbone.jpg}
\end{figure}
	\cite{BIO09}
\end{frame}

\begin{frame}
\frametitle{Struktur der DNA}
\begin{figure}[h]
	\centering
		\includegraphics[width=9cm]{pic/Basen.jpg}
\end{figure}
\centering
	\cite{BIO09}
Stickstoffbasen\\
	\vspace{1cm}
Die drei Bestandteile des Nukleotids geben sowohl die Funktion der DNA, wie auch die Struktur der Doppelhelix vor.\\
\vspace{0.5cm}
Ohne die Elemente P, O, N, C und H ist kein Leben möglich.
\end{frame}

\begin{frame}
\frametitle{Struktur der DNA}
\vspace{0.5cm}
\centering
\begin{minipage}{4.5cm}
\begin{figure}[h]
	\centering
		\includegraphics[width=4cm]{pic/Chemical.jpg}
\end{figure}
\centering
	\cite{BIO09}
	\end{minipage}
	\begin{minipage}{4.5cm}
\begin{figure}[h]
	\centering
		\includegraphics[width=2.3cm]{pic/Ribbon.jpg}
\end{figure}
\centering
	\cite{BIO09}
	\end{minipage}
	\centering
	
	\vspace{0.3cm}
	\small Entscheidend für die Struktur der Doppelhelix sind Wasserstoffbrücken und Stapelwechselwirkungen benachbarter BP.\\
	%%Bei der Denaturierung werden nur H-Brücken aufgebrochen.
	\end{frame}
	
	\begin{frame}
	\frametitle{Stapelwechselwirkungen}
	\small
Phosphatrückgrat: hydrophil \& Basen: hydrophob\\
	$\Rightarrow$ Um das Wasser aus dem Zwischenraum zu verdrängen, "`stapeln"' sich die Basen dicht auf einander.
	\begin{itemize}
	\item Es handelt sich um eine ungerichtete WW
	\item Verringerung der Kontaktfläche zwischen Wasser und lipophiler Schicht \\ $\Rightarrow$ Entropiegewinn
	\item Chemisch durch eine WW der $\pi$-Elektronensysteme der beiden Basen zu verstehen.
	\item Typische Eigenschaften von planaren, (quasi-)aromatischen Strukturen.
	\item Stapel-WW ist sequenzabhängig, am schwächsten ist AT-AT	und GC-GC am stärksten (TD stabiler)
	\end{itemize}
	\begin{block}{}
	Bei der Denaturierung werden zuerst die dominierenden Stapel-WW überwunden, anschließend die schwächeren H-Brücken aufgebrochen.
	\end{block}
	
	\end{frame}
	
	
	
	\begin{frame}
	\poslogo
	\begin{textblock}{1}(0.488,0.005)
   \includegraphics[width=1cm]{pic/hut.png}
  \end{textblock} 
	\frametitle{Watson-Crick-Modell}
	\begin{block}{Ein Durchbruch für die Strukturaufklärung}
	\begin{itemize}
	\item 1953 veröffentlichten Watson und Crick einen kurzen Artikel über die rechtshändige, helikale Struktur der DNA
	\item 1962 erhielten Watson, Crick \& Wilkins für ihr räumliches Modell der DNA den Nobelpreis für Medizin
	\end{itemize}
	\end{block}	
	\vspace{0.2cm}
	\begin{itemize}
	\item Theorien wurden aus Röntgenaufnahmen abgeleitet.
	\item Vor der Veröffentlichung wurden Verwicklungen von 3 Strängen postuliert.
	\end{itemize}
	Watson \& Crick erkannten unteranderem, dass 
	\begin{itemize}
	\item nur A--T und G--C binden.
	\item eine Desoxyribose vorhanden sein muss.
	\item H-Brücken die entscheidende WW sind.
	\end{itemize}
	\end{frame}
