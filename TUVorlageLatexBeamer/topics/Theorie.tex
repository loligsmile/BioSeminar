
\section[Theoretische Modelle]{Die Theorie der Denaturierung}
\setstretch{1}

\subsection{Grundlagen}
\begin{frame}
\frametitle{Grundlagen der Standardmodelle}
Die meisten Modelle bauen auf einem Basismodell ähnlich dem Isingmodell von B. H. Zimm auf (1960).\\
\vspace{0.4cm}
Verwendete Annahmen
\begin{itemize}
\item DNA als 1D Gitter aus N BP
\item H-Brücken zwischen komplementären Basen
\item Hydrophobe/ Stapelwechelwirkung zwischen nächsten Nachbarn
\item Zwei Zustände für BP: gebunden / ungebunden
%\item Berücksichtigung der Konfigurationsentropie und des Disszoziationsgleichgewichts
\end{itemize}
\end{frame}

\subsection{Basenpaar-Modell}
\begin{frame}
\frametitle{Basenpaar-Modell}
\begin{minipage}{5.2cm}
\setstretch{1}
\begin{figure}[h]
	\centering
		\includegraphics[width=5.0cm]{pic/BPModell.jpg}
\end{figure}
\centering
\tiny\cite{GOT83}
\end{minipage}
\begin{minipage}{5.0cm}
\setstretch{1}
\small{
\begin{itemize}
\item Nummerierung der BP vom 5' zum 3' Ende
\item Beschreibung durch einen N-Dimensionalen Vektor $\vec{c}$ 
\item BP: offen 0, geschlossen 1
\end{itemize}
}
\end{minipage}
\vspace{0.2cm}
\setstretch{1}
Zustandsumme $Z_c$ ist das Produkt der Stapilitätsparameter $s_k$ und Loop-Gewichtungsfaktoren $\sigma_l$.
Faktoren werden nach folgenden Regeln verwendet.
\begin{itemize}
\item Gebundenes k-tes BP $\Rightarrow\, s_k$
\item Innerer Loop mit l ungebundenen BP $\Rightarrow\, \sigma_l$
\item Gewichtung 1 für die Enden der Kette (keine freie Energie).
\end{itemize}
\end{frame}

%\begin{frame}
%\frametitle{\large Freie Energie und Wahrscheinlichkeiten}

%%%%%%%%%%%%%%%%%%%%%%%%%%%%%%%%%%%%%%%%%%%%%%%%%%%%%%%%%%%%%%%%%%%%%%%%%%%%%%%%%%%%%%%%%%%%%%%%%%%%%%%%%%%%%%%%%%POSITION DER FOLIE?? ETWAS NICHTSSAGEND?\\
%Freie Energie
%\begin{align*}
%F_c=-\text{R}T\ln(Z_c)
%\end{align*}
%Wahrscheinlichkeit der Konfiguration $c$
%\begin{align*}
%P_c=\frac{Z_c}{\sum Z_c}
%\end{align*}

%Helizität einer Doppelhelix mit $N_c(1)$ gebundenen BP
% \begin{align*}
%\Theta= \frac{\sum N_c(1) Z_c}{\sum Z_c}
%\end{align*}
%\end{frame}


\subsection{Stacking-Modell}
\begin{frame}
\frametitle{Stacking-Modell}
\small Die dominante, strukturgebende WW der DNA ist die Stacking-WW \\
\centering{ $\Rightarrow$ anpassen des Modells.}\\
\begin{minipage}{5.2cm}
\begin{figure}[h]
	\centering
		\includegraphics[width=5.0cm]{pic/STModell.jpg}
\end{figure}
\centering
\tiny\cite{GOT83}
\end{minipage}
\begin{minipage}{5.0cm}
\small{
\begin{itemize}
\item Jedem BP Doublet wird ein Stabilitätsparamter zugewiesen
\item Beschreibung durch einen N-1-Dimensionalen Vektor $\vec{c}$
\item WW zwischen Doublet $\hat{=}$ stacked
\item stacked 1 / unstacked 0
\end{itemize}
}
\end{minipage}

\begin{flushleft}
Regeln der Zustandssumme können übernommen werden, mit
\begin{itemize}
\item $l$ = \# unstacked Doublets= \# offene BP + 1
\item $k$ = Doublet aus k-tem und (k+1)sten BP
\end{itemize}
\end{flushleft}
\end{frame}

\subsection{Das Parameterproblem}
\begin{frame}
\frametitle{Das Parameterproblem}

\textbf{Stabilitätsparameter} des Basenpaars MN
\begin{align*}
s_k=s_{MN}=\exp\left(-\frac{\Delta H_{MN}-T\Delta S_{MN}}{\text{R}T}\right)
\end{align*}
\begin{itemize}
\item Die Entropie $\Delta S_{MN}$ und Enthalpie $\Delta H_{MN}$ werden als konstant angenommen
\item Kalorische Messungen zeigten $\Delta S_{MN}\approx \Delta S \approx \text{const}$
\item Relevanter Parameter ist die Schmelztemperatur $T_{MN}=\frac{\Delta H_{MN}}{\Delta S}$
\end{itemize}
\vspace{0.4cm}
Vereinfachende Annahme $T_{MN}=T_{NM}$, und $T_{MN}=(T_M+T_N)/2$\\
$\Rightarrow$ Zwei unabhängige Parameter, da $T_A=T_T$ und $T_G=T_C$\\
\vspace{0.3cm}
Bestimmung von $T_A$ und $T_G$ aus Modell DNA mit 100\% AT / GC BP.
\end{frame}



\begin{frame}
\frametitle{Das Parameter Problem}
\textbf{Loop gewichtende Faktoren}
\begin{itemize}
\item müssen von der Loop-Entropie abhängen
\item sollten für innere Loops < 1 sein (weniger Freiheitsgrade)
\item Fallende Funktion, aber langsamer als $\exp(-l)$
\item müssen mit der Ionenstärke abnehmen
\item müssen mit der experimentell beobachten Steifheit der Helix einhergehen
\end{itemize}
\vspace{0.3cm}
\centering
$\Rightarrow$ Kein Beweis für die genaue funktionale Form von $\sigma_l$\\
Meinungen gehen stark auseinander!
\vspace{0.2cm}
\textcolor{YellowGreen}{
\small{
\begin{align*}
\sigma_l=\sigma_0 \lambda_l \epsilon_l \; \text{?} \qquad \sigma_l=\sigma_0\Psi_l\cdot\frac{l\cdot c_l}{c_\infty}\; \text{?}\qquad \sigma_l=\sigma_0 l^{-c}\;\text{?}\qquad \sigma_l=\sigma_0 (l-d)^{-c}\;\text{?}
\end{align*}}}
\end{frame}


\begin{frame}
\frametitle{Das Parameter Problem}
Bereits bei diesem einfachen Modell mussten \textcolor{YellowGreen}{sehr viele Annahmen} gemacht werden. \\
\vspace{0.2cm}
Bei komplexeren Betrachtungen, welche beispielsweise
\begin{itemize}
\item Die Zustandssumme in eine Interne und Externe aufspalten.
\item Die Abhängigkeit von Molekülmasse und Form berücksichtigen. 
\item Heterogenitäten der Sequenz berücksichtigen. 
\item Änderung von Translations- und Rotationsfreiheitsgraden betrachten.
\item Die Loop-Entropie als zustätzlichen Parameter einführen.
\end{itemize}
\vspace{0.2cm}
werden die Meinungsverschiedenheiten und Unsicherheiten der Theorien nur größer.
\end{frame}

\subsection{Vergleich Theorie \& Experiment}
\begin{frame}
\frametitle{Vergleich mit Messdaten}
\small Unterscheidung zwischen langer >1000 BP  und kurzer <600 BP DNA ist nötig.\\
\begin{minipage}{5.2cm}
\begin{figure}[h]
	\centering
		\includegraphics[width=6.0cm]{pic/EXPlang.jpg}
\end{figure}
\centering
\tiny\cite{WAR85}
\end{minipage}
\hfill
\begin{minipage}{5.2cm}
\begin{figure}[h]
	\centering
		\includegraphics[width=3.5cm]{pic/EXPkurz.jpg}
\end{figure}
\centering
\tiny\cite{WAR85}
\end{minipage}
\centering
Woran liegen die Abweichungen??
\end{frame}

\begin{frame}
\frametitle{Hysterese}
Mögliche Ursache: Theorien berechnen Gleichgewichtsschmelzkurven\\
\vspace{0.2cm}
\begin{block}{Hysterese- Effekt}
Teilweise denaturierte DNA zeigt Irreversibilitäten in ihren Schmelzkurven,
 wenn mit der gleichen Rate gekühlt wird, wie zuvor geheizt.
\end{block}
\vspace{0.2cm}
Gewöhnlich wird mit $v=0,1-0,2 ^\circ$C / min geheizt.\\
Für Gleichgewichtsschmelzkurven muss gelten 
\begin{align*}
t_{\text{rel}}<t_\text{exp}=\frac{\Delta T_m}{v}
\end{align*}
Bsp. 100-300 BP, $\Delta T_m\approx 0,3 ^\circ$C und  $v=0,002 ^\circ$C /min $\Rightarrow$ $t_{\text{rel}}<150$s\\
\vspace{0.3cm}
\centering Biomoleküle können sehr langsam sein.
\end{frame}

%%%%%%%%%%%%%%%%%%Zusätzlicher Teil zu möglichen Ursachen: Stem Bildung, Heterogene WW, Versuch der Verbesserung: systematische Paramtervariiation (iterativ), objektives Kriterium was offen und geschlossen heißt (reicht eine Veränderung der elektronischen Konfiguration schon aus?), was gute und schlechte Übereinstimmung heißt, einfluss chemischer Umgebung, usw...



%\subsection{Statistischer Ansatz}
%\begin{frame}
%\frametitle{Statistischer Ansatz}
%Theorie nach Peyrard \& Bishop mit folgenden Annahmen
%\begin{itemize}
%\item Auslenkung der BP aus der Gleichgewichtslage entlang H-Brücken
%\item Zwei Freiheitsgrade $u_n$ und $v_n$ \\
%$\Rightarrow$ $x_n=\frac{1}{\sqrt{2}}(u_v+v_n)^2$ und $y_n=\frac{1}{\sqrt{2}}(u_n-v_n)^2$
%\item Potential der H-Brücken wird durch das Morse-Potential beschrieben
%\item Harmonische Kopplung benachbarter Basen
%\item Vernachlässigen von Inhomogenitäten
%\end{itemize}
%\begin{align*}
%\mathcal{H}=\sum_{n} \frac12 m(\dot{x}_n^2+&\dot{y}_n^2)+ \frac12 k[(x_n-x_{n-1})^2+(y_n-y_{n-1})^2]+V(y_n)\\
%&V(y_n)=D\cdot \left\{\exp(-a\sqrt{2} y_n)-1\right\}^2
%\end{align*}
%\end{frame}
%
%
%
%\begin{frame}
%\frametitle{Statistischer Ansatz}
%Kanonische Zustandssumme faktorisiert
%\begin{align*}
%Z=&\int_{-\infty}^{\infty} \prod_{n=1}^{N} \text{d}x_n\text{d}y_n \text{d}p_{{x}_n} \text{d}p_{{y}_n} \exp(-\beta \mathcal{H}(x_n,y_n))\\
%\Rightarrow Z=&(Z_{x_{1}}\cdot Z_{y_{1}}\cdot Z_{p_{x_1}}\cdot Z_{p_{y_1}})^N
%\end{align*}
%Drei der Integrale sind einfach zu lösen
%\begin{align*}
%Z_{p_{x_1}}=Z_{p_{y_1}}=\sqrt{2\pi m k_B T} \qquad \& \qquad Z_{x_1}=\sqrt{\frac{2\pi}{\beta k}}
%\end{align*}
%Das Integral $Z_{y_1}$ lässt sich im TD-Limes $N\rightarrow \infty$ exakt lösen, in dem die Eigenfunktionen und Eigenwerte eines Transferintegral verwendet werden.
%\end{frame}
%
%
%
%\begin{frame}
%\frametitle{Statistischer Ansatz}
%Möglichkeit der Berechnung der freien Energie und Erwartungswerten.\\
%\vspace{0.2cm}
%Beispielsweise die mittlere Streckung der H-Brücke
%\begin{align*}
%<y_m>=\frac1Z \int \prod y_m \exp(-\beta  \mathcal{H}) \text{d}x_n\text{d}y_n \text{d}p_{{x}_n} \text{d}p_{{y}_n}
%\end{align*}
%\vspace{0.5cm}
%Aber auch hier gibt es Schwierigkeiten
%\begin{itemize}
%\item Die Parameter können nicht hinreichend genau bestimmt werden
%\item Zwei Freiheitsgrade scheinen nicht zur Beschreibung auszureichen
%\item Nichtlinearitäten wurden nicht berücksichtigt
%\item Lösen von sensitiven nichtlinearen Systemen nötig
%\end{itemize}
%\end{frame}




\section{Phasenübergang?!}

\subsection{Grundlagen}
 \begin{frame}
\frametitle{Phasenübergang??}
\begin{block}{}
1D statistisch-mechanische Systeme mit kurzreichweitigen WW zeigen keinerlei Phasenübergänge für T>0
\end{block}
\vspace{0.5cm}
\begin{itemize}
\item Bei beinahe 1D Systemen und
\item Systemen mit langreichweitiger WW
\end{itemize}
können jedoch Phasenübergänge beobachtet werden.\\
\vspace{0.5cm}
\centering
$\Rightarrow$ Ursache und Ordnung des Phasenübergangs der DNA?
\end{frame}

\begin{frame}
\frametitle{Phasenübergang!!}
\begin{block}{Vorgehen}
\setstretch{1.5}
\vspace{0.3cm}
\begin{itemize}
\item Ableiten der Zustandssumme aus der freien Energie
\item Bedeutung der Singularität der großkanonischen Zustandssumme
\item Ansatz für das Necklace - Modell und dessen Bedeutung
\item Schlussfolgerung der Ordnung des Übergangs
\end{itemize}
\end{block}
\setstretch{1.0}
\end{frame}

\begin{frame}
\small
\vspace{0.9cm}
Freie Energie pro Längeneinheit $f(T)$
\begin{align*}
f(T)=-\lim_{n\rightarrow\infty} \frac1n \ln(Z_n(T))\\
\Rightarrow Z_n(T)=\exp(- f(T) n)
\end{align*}
\vspace{0.3cm}
Für die großkanonische Zustandssumme gilt
\begin{align*}
G(z,T)=\sum^{\infty}_{n=0}z^n\, Z_n(T)
\end{align*}
mit der Fugazität $z=\exp(\beta \mu)$.\\
\vspace{0.3cm}
Die Reihe konvergiert für $z^nZ_n(T)<1\,\Rightarrow$Kleinste Singularität bei $z_0^nZ_n(T)=1$ 
\begin{align*}
z_0=\exp( f(T))\\
f(T)=\ln(z_0)
\end{align*}
\end{frame}

\begin{frame}
\frametitle{\large Ordnung des Phasenübergangs}
\small
%Die niedrigste Singularität gibt die freie Energie des Systems vor.\\
Gibt es mehrere Singularitäten kann es zum Phasenübergang kommen.
Singularitäten sind Funktionen z.B  $z_0(T)$. Der Zweig der freien Energie kann gewechselt werden, da das System eine minimal freie Energie bevorzugt.\\
\centering $\Rightarrow$ Punkt des Phasenübergangs
\vspace{0.3cm}
\begin{figure}[h]
	\centering
		\includegraphics[width=0.50\textwidth]{pic/energie.jpg}
\end{figure}
\centering
 \textcolor{blue}{$f_1(T)$} Phasenübergang 1. Ordnung\\
 \textcolor{red}{$f_1'(T)$} Kontinuierlicher Phasenübergang 
\end{frame}


\subsection{Das Necklace - Modell}
\begin{frame}
\frametitle{\large Anwendung des Necklace - Modells}
\begin{figure}[h]
	\centering
		\includegraphics[width=0.80\textwidth]{pic/Necklace.JPG}
\end{figure}
\centering{\small\cite{FIS84}}\\
\vspace{0.3cm}
\small
Mit den kanonischen Zustandssummen $Q_n$ ergibt sich
\begin{align*}
G_A(z)=\sum_n Q_n^A z^n \qquad \qquad G_B(z)=\sum_n Q_n^B z^n
\end{align*}
Unter der Annahme, das die Ränder immer geschlossen sind gilt
\begin{align*}
G(z)=G_A+G_A\nu G_B\nu G_A+G_A\nu G_B\nu G_A \nu G_B \nu G_A+...
\end{align*}
mit $\nu$ als Gewichtungsfaktoren der Vertices
\end{frame}

\begin{frame}
Wenn nun z hinreichend klein ist, konvergiert die Reihe zu
\begin{align*}
G(z)=\frac{G_A(z)}{1-\nu^2 G_A(z)G_B(z)}
\end{align*}

Außer bei den Singularitäten von
\begin{itemize}
\item  $G_A(z)$ und $G_B(z)$ 
\item $\nu ^2 G_B(z)=\frac{1}{G_A(z)}$
\end{itemize}
\vspace{0.3cm}
Die Ordnung des Phasenübergangs erhält man durch einen konkreten Ansatz für $Q_A$ und $Q_B$.
\end{frame}



\begin{frame}
\frametitle{Der Ansatz}
\small Geschlossener Zustand: fester Energiebetrag pro Bindung, kein Entropiebeitrag
\begin{align*}
Q_A^n=u^n\qquad \qquad u=\exp(-\beta \epsilon)
\end{align*}

\vspace{0.15cm}
\small Offener Zustand: Keine Bindungsenergie, rein entropischer Beitrag
\begin{align*}
Q_B^n=\frac{q_0 w^n}{n^\Psi} \qquad \qquad w=\exp(-\sigma_o(T))
\end{align*}
\begin{itemize}
\item Vergleich des Ansatzes mit dem BP-Modell:\\
Die Faktoren $u$ und $w$ spiegeln den Stabilitätsparameter $s_k$ wider.\\
Das Necklace-Modell berücksichtigt keine Sequenzabhängigkeit $\Rightarrow s_k=\text{C}$.\\
$\sigma_l \,=\, \sigma_0 l^{-c}\; \hat{=}\; q_o n^{-\Psi}$ beschreibt die Loop-Entropie.\\
\item $\Psi$ ermöglicht die Anpassung der Möglichkeiten an gegebene Bedingungen\\
Loop: Random-Walk mit Rückkehr und Selbstvermeidung.
\end{itemize}

\end{frame}

\begin{frame}
\frametitle{\large Ordnung des Phasenübergangs}
\begin{align*}
G_B(z)=\sum_n \frac{q_o (wz)^n}{n^\Psi}
\end{align*}
\centering
Das Konvergenzverhalten bei $wz\rightarrow 1$ hängt von $\Psi$ ab.\\
\vspace{0.3cm}
Untersuchung: Schnittpunkt der Singularitäten von $G(z)$ bei $wz\rightarrow1$

\vspace{0.3cm}
\begin{block}{Zusammenfassung der Ergebnisse}
\begin{itemize}
\item $\Psi \leq1\qquad\qquad$$G_B(z)$ und $G(z)$ divergieren
\item $1<\Psi<2\qquad$  Kontinuierlicher Phasenübergang
\item $\Psi>2\qquad\qquad$ Phasenübergang 1. Ordnung
\end{itemize}
\end{block}
\vspace{0.3cm}
\end{frame}


\begin{frame}
\frametitle{Zurück zur DNA..}
Für einen Random-Walk in 3D, welcher zurückkehrt, ergibt sich $\Psi=1,5$.\\
Jedoch steigt $\Psi$ unter Berücksichtigung von
\begin{itemize}
\item Selbstvermeidung des Loops
\item Vermeidung benachbarter Loops
\item Sequenzabhängigkeit der Stabilität
\item Stem-Bildung
\end{itemize}
\begin{block}{}
Für DNA ergibt sich ein Phasenübergang 1. Ordnung
\end{block}
Der Faktor $n^{-\Psi}$ kann als langreichweitige WW verstanden werden.
\begin{itemize}
\item Dies erklärt die Existenz eines Phasenübergangs dieses "`1D"' Systems
\item Zeigt die Grenzen eines Ising-artigen Modells
\end{itemize}
\end{frame}